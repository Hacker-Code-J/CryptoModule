%%%%%%%%%%%%%%%%%%%%%%%%%%%%%%%%%%%%%%%%%%%%%%%%%%%%%%%%%%%%%%%%%%%%%%
% Quantum-Safe Cryptography Report: Building & Installing New Signature Algorithm
%%%%%%%%%%%%%%%%%%%%%%%%%%%%%%%%%%%%%%%%%%%%%%%%%%%%%%%%%%%%%%%%%%%%%%

\documentclass[11pt,a4paper]{report}

%-------------------------
% Package Imports
%-------------------------
\usepackage[margin=1in]{geometry}            % Page margins
\usepackage{graphicx}                        % Images and logos
\usepackage{titlesec}                        % Custom section formatting
\usepackage{fancyhdr}                        % Headers and footers
\usepackage{setspace}                        % Line spacing
\usepackage{hyperref}                        % Hyperlinks
\usepackage{enumitem}                        % Customized lists
\usepackage{lmodern}                         % Enhanced fonts
\usepackage{xcolor}                          % Color definitions
\usepackage{array}                           % Table formatting
\usepackage{listings}                        % For code listings

\usepackage{tikz}
\usetikzlibrary{arrows.meta,calc,positioning}
\input{lib/crypto.symbols}
\usetikzlibrary{trees}
\usepackage{pgfplots}
\pgfplotsset{compat=1.17}

\usepackage{amsmath, amssymb, amsfonts, amsthm, mathtools}
\usepackage{commath}
\usepackage{algorithm, algorithmic}
\usepackage{tcolorbox}
\tcbset{colback=white, arc=5pt}
\newcommand{\defbox}[2][]{%
	\begin{tcolorbox}[colframe=black, title={\color{white}\bfseries #1}]
		#2
	\end{tcolorbox}
}

\usepackage{adjustbox}

% Fonts
\usepackage[T1]{fontenc}
\usepackage[utf8]{inputenc}
\usepackage{newpxtext,newpxmath}
\usepackage{sectsty}

%%%%%%%%%%%%%%%%%%%%%%%%%%%%%%%%%%%%%%%%%%%%%%%%%%%%%%%%%%%%%%%%%%%%%%
% Theorem, Definition, and Remark Environments
\newtheorem{theorem}{Theorem}[chapter]
\newtheorem{lemma}[theorem]{Lemma}
\newtheorem{proposition}{Proposition}[chapter]
\newtheorem{corollary}[theorem]{Corollary}
\newtheorem{problem}{Problem}[chapter]

\newtheoremstyle{definitionstyle} % Name of the style
{3pt} % Space above
{3pt} % Space below
{} % Body font
{} % Indent amount
{\bfseries} % Theorem head font
{.} % Punctuation after theorem head
{2.5mm} % Space after theorem head
{} % Theorem head spec
\theoremstyle{definitionstyle}
\newtheorem*{observation}{\textcolor{Magenta}{Observation}}
\newtheorem{definition}{Definition}[chapter] % Definition shares the counter with theorem
\newtheorem{example}{Example}[chapter] % Example shares the counter with theorem
\newtheorem{exercise}{{Exercise}}[chapter] % Example shares the counter with theorem
\newtheorem{remark}{Remark}[chapter] % Remark shares the counter with theorem
%\newtheorem*{note}{Note}[chapter]

%%%%%%%%%%%%%%%%%%%%%%%%%%%%%%%%%%%%%%%%%%%%%%%%%%%%%%%%%%%%%%%%%%%%%%
% Custom Commands for Notation and Symbols
\newcommand{\bit}{\{0,1\}}
\newcommand{\GF}{\operatorname{GF}}
\newcommand{\xor}{\oplus}
\newcommand{\Cons}{\operatorname{Cons}}
\newcommand{\E}{\operatorname{E}}
\newcommand{\Einv}{\operatorname{E}^{-1}}
\newcommand{\PRP}{\operatorname{PRP}}
\newcommand{\KeySpace}{\{0,1\}^k}
\newcommand{\BlockSpace}{\{0,1\}^n}
\newcommand{\Prob}{\mathbb{P}}
\newcommand{\Ex}{\mathbb{E}}
\newcommand{\Var}{\operatorname{Var}}
\newcommand{\supp}{\operatorname{supp}}
\newcommand{\floor}[1]{\lfloor #1 \rfloor}
\newcommand{\ceil}[1]{\lceil #1 \rceil}
\newcommand{\AES}{\textsf{AES}}


%-------------------------
% Custom Fonts and Colors
%-------------------------
%\allsectionsfont{\sffamily\bfseries}         % Sans-serif, bold section titles

% Define a rule command for the cover page
\newcommand{\HRule}{\rule{\linewidth}{0.5mm}}

%-------------------------
% Listing Settings
%-------------------------
\lstset{
	basicstyle=\footnotesize\ttfamily,
	breaklines=true,
	frame=single,
	columns=fullflexible,
	captionpos=b,
	numbers=left
}

%-------------------------
% Header & Footer Settings
%-------------------------
\pagestyle{fancy}
\fancyhf{}
\fancyhead[L]{\leftmark}
\fancyhead[R]{\thepage}
\renewcommand{\headrulewidth}{0.4pt}


\usepackage{background}
\backgroundsetup{
	scale=1,  % Scale the image to fit the page
	color=black,  % Keep the image color as is
	opacity=.1,  % Fully opaque background
	angle=0,  % No rotation
	%	position=current page.south,  % Align to bottom of the page
	vshift=1cm,  % Adjust vertical alignment if needed
	%	hshift=0cm,  % Adjust horizontal alignment if needed
	contents={}
}
\newcommand{\setPageBackground}{
	\backgroundsetup{
		contents={\includegraphics[scale=4]{images/school-logo}
			%		contents={\includegraphics[width=\paperwidth,height=\paperheight]{images/school-logo}
			} % Replace with your image
		}
	}

%-------------------------
% Title Page Setup
%-------------------------
\begin{document}
\begin{titlepage}
	\setPageBackground
	\centering
	\vspace*{1cm}
	
	% Company logo (optional)
%		\includegraphics[width=0.3\textwidth]{logo.png}\par\vspace{1cm}
	
	\HRule \\[0.4cm]
	{\Huge \sffamily \bfseries Cryptographic S/W Modules with C}\\[0.4cm]
	\HRule \\[1.5cm]
	
	{\Large Design, Implementation, and Integration of Core Crypto Modules}\\[0.5cm]
	{\normalsize Secure, Efficient, High-Performance Cryptographic Software Modules}\\[2cm]
	
	\begin{flushright}
		\LARGE {\bfseries Ji, Yong-hyeon} \\ 
		{\Large\texttt{hacker3740@kookmin.ac.kr}} \\[1.5cm]
		\large Department of Cyber Security \\
		Kookmin University \\ [1cm]
		\today
	\end{flushright}
	\vfill
\end{titlepage}

%-------------------------
% Table of Contents
%-------------------------
\tableofcontents
\clearpage

%-----------------------------------------------------------------------
%  CHAPTER 1: PROJECT OVERVIEW
%-----------------------------------------------------------------------
\chapter{Project Overview}
%\section{Purpose and Scope}

This document provides a comprehensive guide to the design, implementation, and integration of cryptographic modules written in C (sometimes assembly). 
%It is intended for developers, security engineers, and maintainers who need to understand the internal structure, coding guidelines, and best practices for working with these modules. By referencing both the C and ASM sources, readers will learn how to optimize cryptographic routines, ensure consistent interfaces, and maintain robust security properties throughout the system.
\ \\
\ \\ \noindent
\textbf{Key Objectives:}
\begin{itemize}
	\item Describing the cryptographic primitives and algorithms\par 
	(block ciphers, hash functions, MACs, signature algorithms, etc.).
	\item Explaining the structure of the source files and headers.
	\item Providing guidelines for building, testing, and integrating these modules into larger software systems.
%	\item Highlighting performance optimization strategies in ASM.
\end{itemize}

\section{Directory Structure}
\begin{center}
\begin{tikzpicture}[%
	grow via three points={one child at (0.5,-0.7) and
		two children at (0.5,-0.7) and (0.5,-1.4)},
	edge from parent path={(\tikzparentnode.south) |- (\tikzchildnode.west)}]
	\tikzstyle{every node}=[draw=black,thick,anchor=west, 
	minimum width=2.25cm, minimum height=.6cm]
	\tikzstyle{selected}=[draw=red,fill=red!30]
	\tikzstyle{optional}=[dashed,fill=gray!50]
	\node {CryptoModule/}
	child { node {bin} }
	child { node {build} 
		child { node {\texttt{*.o}} }
		child { node {\texttt{*.d}} }
	}
	child [missing] {}		
	child [missing] {}
	child [missing] {}
	child { node[fill=blue!30, draw=blue] {include}
		child { node {block\_cipher}
			child { node {\texttt{block\_cipher.h}} }
			child { node {\texttt{block\_cipher\_aes.h}} }
			child { node {$\cdots$} }
		}
		child [missing] {}	
		child [missing] {}		
		child [missing] {}
		child { node {mode}
			child { node {\texttt{mode.h}} }
			child { node {$\cdots$} }
		}	
		child [missing] {}
		child [missing] {}
		child { node {$\cdots$} }
		child { node {\texttt{api.h}} }
	}	
	child [missing] {}	
	child [missing] {}
	child [missing] {}
	child [missing] {}	
	child [missing] {}	
	child [missing] {}	
	child [missing] {}	
	child [missing] {}	
	child [missing] {}	
	child [missing] {}	
	child { node[fill=green!50, draw=green!75!black] {src/}
		child { node {block\_cipher}
			child { node {\texttt{block\_cipher\_factory.c}} }
			child { node {\texttt{block\_cipher\_aes.c}} }
			child { node {$\cdots$} }
		}
		child [missing] {}	
		child [missing] {}		
		child [missing] {}
		child { node {mode}
			child { node {\texttt{mode\_factory.h}} }
			child { node {$\cdots$} }
		}	
		child [missing] {}
		child [missing] {}
		child { node {$\cdots$} }
		child { node {\texttt{cryptomodule\_core.h}} }
		child { node {\texttt{main.c}} }
	}
	child [missing] {}
	child [missing] {}	
	child [missing] {}	
	child [missing] {}	
	child [missing] {}	
	child [missing] {}	
	child [missing] {}	
	child [missing] {}	
	child [missing] {}	
	child [missing] {}	
	child [missing] {}	
	child { node[fill=cyan!50, draw=cyan] {tests/} }
	child { node[fill=orange!50, draw=orange] {\texttt{Makefile}} }
	;
\end{tikzpicture}
\end{center}
The code base is organized to reflect modular cryptographic primitives and functionality. The main directories and their purposes are outlined below:

\begin{itemize}
	\item \textbf{include/}: Contains all public headers for cryptographic modules.
	\begin{itemize}
		\item \texttt{cryptomodule/block/}: Headers for block cipher implementations (e.g., AES).
		\item \texttt{cryptomodule/mode/}: Headers for modes of operation (CBC, CTR, GCM, etc.).
		\item \texttt{cryptomodule/rng/}: Headers for random number generators.
		\item \texttt{cryptomodule/hash/}: Headers for hash functions (e.g., SHA-256, SHA-512).
		\item \texttt{cryptomodule/mac/}: Headers for message authentication codes (e.g., HMAC).
		\item \texttt{cryptomodule/kdf/}: Headers for key derivation functions (PBKDF, HKDF, etc.).
		\item \texttt{cryptomodule/keysetup/}: Headers for key exchange primitives (ECDH).
		\item \texttt{cryptomodule/sign/}: Headers for signature algorithms (ECDSA, RSA, etc.).
	\end{itemize}
	
	\item \textbf{src/}: Contains the corresponding C/ASM source files for each cryptographic category.
	\begin{itemize}
		\item \texttt{block/}, \texttt{mode/}, \texttt{rng/}, \texttt{hash/}, \texttt{mac/}, \texttt{kdf/}, \texttt{keysetup/}, \texttt{sign/}
	\end{itemize}
	
	\item \textbf{tests/}: Houses unit tests and integration tests for all cryptographic modules.
	
	\item \textbf{Makefile}: Defines how to build and link the libraries and tests. Contains flags for C and ASM code.
	
	\item \textbf{README.md}: Provides a high-level overview of the project, including build instructions and usage examples.
\end{itemize}

\subsection{Hierarchy and Relationships}

Each functional category (block cipher, hash, etc.) is encapsulated in its own subdirectory to keep code organized and maintainable. Corresponding header files in \texttt{include/cryptomodule/} expose the public API, while the implementations in \texttt{src/} include both C and, where appropriate, ASM files for optimized routines.

\section{Build Tools and Dependencies}

A standard Unix-like build environment is assumed, with the following tools and dependencies required:

\begin{itemize}
	\item \textbf{Compiler} (e.g., \texttt{gcc} or \texttt{clang}) with support for assembling inline or separate ASM files.
	\item \textbf{Make} (GNU Make) to use the provided \texttt{Makefile}.
	\item \textbf{CMake (optional)}: Some teams prefer CMake-based workflows; a \texttt{CMakeLists.txt} can also be maintained for cross-platform compatibility.
	\item \textbf{Perl/Python (optional)}: May be required for certain test scripts, code generation, or performance analysis scripts.
	\item \textbf{OpenSSL (optional)}: Useful for comparing test vectors or for using the system’s cryptographic library as a reference.
\end{itemize}

When building the library, you can enable or disable specific optimizations or algorithms by modifying the \texttt{Makefile} (or \texttt{CMakeLists.txt}, if you choose to add one). For instance, enabling ASM routines for AES might require additional flags like:
\begin{lstlisting}
	CFLAGS += -march=native -maes
\end{lstlisting}
depending on your target CPU capabilities.

\subsection{Environment Configuration}

Before compiling, ensure that your development environment is set up with the correct paths. For instance:
\begin{lstlisting}
	export CC=gcc
	export AS=nasm     # or another assembler if preferred
	export CFLAGS="-O2 -Wall -Wextra"
\end{lstlisting}
Adjust these variables as needed based on your local toolchain and performance requirements.

\section{Coding Guidelines}

All C code should follow a consistent style (e.g., \texttt{K\&R} or \texttt{LLVM} style) with adequate comments explaining the purpose and usage of functions. Inline ASM or standalone ASM files should use readable label names, and macros must be well-documented to clarify any platform-specific instructions.

Furthermore, each function in the cryptographic modules should include:
\begin{itemize}
	\item \textbf{Parameter validations}: Ensure pointers are not \texttt{NULL}, lengths are within expected ranges, etc.
	\item \textbf{Error handling}: Return clear error codes and avoid silent failures.
	\item \textbf{Security considerations}: Erase sensitive data buffers immediately after use to prevent leakage.
\end{itemize}

\section{Security and Maintenance Policies}

Because cryptographic libraries are critical to overall system security, the project maintains strict policies regarding:
\begin{itemize}
	\item \textbf{Patch review}: All code changes are peer-reviewed to detect potential vulnerabilities or performance regressions.
	\item \textbf{Regular audits}: Scheduled internal and external audits are conducted to verify compliance with best security practices.
	\item \textbf{Versioning and backward compatibility}: Each stable release is tagged in version control, with major version increments for breaking changes.
\end{itemize}

\chapter{Cryptographic Module}

%-----------------------------------------------------------------------
%  CHAPTER 2: BLOCK CIPHER MODULES (Placeholder)
%-----------------------------------------------------------------------
\section{Block Cipher Modules}
% Detailed content about supported block ciphers, usage examples, ASM optimizations

%-----------------------------------------------------------------------
%  CHAPTER 3: MODES OF OPERATION (Placeholder)
%-----------------------------------------------------------------------
\section{Modes of Operation}
% Detailed content about CBC, CTR, GCM, etc.

%-----------------------------------------------------------------------
%  CHAPTER 4: RANDOM NUMBER GENERATOR (Placeholder)
%-----------------------------------------------------------------------
\section{Random Number Generator}
% Implementation details, entropy sources, usage examples

%-----------------------------------------------------------------------
%  CHAPTER 5: HASH FUNCTIONS (Placeholder)
%-----------------------------------------------------------------------
\section{Hash Functions}
% Implementation details, supported hashes, ASM notes

%-----------------------------------------------------------------------
%  CHAPTER 6: MESSAGE AUTHENTICATION CODES (Placeholder)
%-----------------------------------------------------------------------
\section{Message Authentication Codes}
% Content about HMAC, CMAC, usage examples, best practices

%-----------------------------------------------------------------------
%  CHAPTER 7: KEY DERIVATION FUNCTIONS (Placeholder)
%-----------------------------------------------------------------------
\section{Key Derivation Functions}
% PBKDF, HKDF, usage in password hashing or key expansions

%-----------------------------------------------------------------------
%  CHAPTER 8: KEY EXCHANGE (Placeholder)
%-----------------------------------------------------------------------
\section{Key Exchange}
% ECDH, DH, usage details, integration

%-----------------------------------------------------------------------
%  CHAPTER 9: SIGNATURE ALGORITHMS (Placeholder)
%-----------------------------------------------------------------------
\section{Signature Algorithms}
% ECDSA, RSA, usage patterns, performance tips

%-----------------------------------------------------------------------
%  CHAPTER 3: BUILD AND INTEGRATION (Placeholder)
%-----------------------------------------------------------------------
\chapter{Build and Integration}
% Makefile overview, linking instructions, library creation

%-----------------------------------------------------------------------
%  CHAPTER 4: TESTING (Placeholder)
%-----------------------------------------------------------------------
\chapter{Testing}
% Unit tests, integration tests, reference vectors

%%-----------------------------------------------------------------------
%%  CHAPTER 5: FAQ / TROUBLESHOOTING (Placeholder)
%%-----------------------------------------------------------------------
%\chapter{FAQ / Troubleshooting}
%% Common errors, solutions, performance tuning

\newpage
\appendix
%-----------------------------------------------------------------------
%  CHAPTER 13: APPENDICES (Placeholder)
%-----------------------------------------------------------------------
\chapter*{Appendices}
% Glossary, references, external documentation
\end{document}
