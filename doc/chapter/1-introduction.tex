\chapter{Introduction}
This document details the design, implementation, and analysis of two C applications that leverage Elliptic Curve Digital Signature Algorithm (ECDSA) to ensure data integrity and non-repudiation in big data processing environments.\par
The first application, `\texttt{filecheck\_ecc}', provides command-line-based file signing and verification. The second application, a `\texttt{logtool\_ecc}' and `\texttt{logaggregator\_ecc}', establishes a framework for creating cryptographically signed, tamper-evident log streams suitable for distributed systems. Both systems are built upon the modular \texttt{libecc}\cite{libecc-github} cryptographic library, utilizing the SECP256R1 curve and SHA-256 (or SHA3-256) hashing algorithm.\par
This document presents the system architecture, a comprehensive developer API guide, performance benchmarks, and security considerations for both applications, demonstrating a practical approach to embedding strong cryptographic guarantees into data-intensive workflows.
\section{Project Scope and Objectives}

This report presents two prototype applications built around ECDSA on the NIST P-256 curve:  
\begin{enumerate}
	\item A \emph{File Integrity Checker} that signs and verifies large files via streaming SHA-256 and chunked ECDSA operations.  
	\item A \emph{Tamper-Evident Log Aggregator} for real-time signing of high-volume log entries with rotating output chunks.  
\end{enumerate}
The primary objectives are to demonstrate scalable ECDSA integration, evaluate performance trade-offs, and document design patterns for secure big data processing.

This project aims to address the aforementioned problem by designing and implementing a framework of two C-based applications built upon the \texttt{libecc} library. The scope is focused on providing practical, command-line-driven tools that can be easily scripted and integrated into existing data pipelines. The primary objectives are as follows:
\begin{enumerate}
	\item \textbf{Develop a Secure File Integrity Tool:} To create a robust utility, \texttt{filecheck\_secure}, for signing and verifying the integrity and authenticity of large data files, such as datasets, archives, and backups.
	\item \textbf{Develop a Secure Logging Framework:} To build a suite of applications, \texttt{logtool} and \texttt{logaggregator}, capable of generating cryptographically signed, tamper-evident log entries and processing them in a streaming fashion.
	\item \textbf{Utilize Industry-Standard Cryptography:} To implement all functionalities using the Elliptic Curve Digital Signature Algorithm (ECDSA) with the widely adopted SECP256R1 (NIST P-256) curve and the SHA-256 hash algorithm.
	\item \textbf{Provide a Clear Developer Guide:} To document the core API functions used from the underlying cryptographic library, enabling other developers to extend the tools or build new secure applications.
\end{enumerate}

\section{Summary of Contributions}

\begin{itemize}
	\item \textbf{Modular C implementations} of streaming ECDSA signing and verification, optimized for large files and logs.  
	\item \textbf{Design patterns} for chunked hashing, memory-bounded processing, and rotation of signed data segments.  
	\item \textbf{Performance evaluation} and best practices for deploying ECDSA in high-throughput Linux environments.  
\end{itemize}

\section{Report Structure}

The remainder of this report is organized as follows:

\begin{itemize}
	\item \textbf{Section 2:} Standards \& Specifications (FIPS 186-5 ECDSA, P-256 parameters).  
	\item \textbf{Section 3:} System Architecture and Design Patterns.  
	\item \textbf{Section 4:} File Integrity Checker—implementation details and benchmarks.  
	\item \textbf{Section 5:} Tamper-Evident Log Aggregator—pipeline design and throughput results.  
	\item \textbf{Section 6:} Performance \& Security Considerations, including key management.  
	\item \textbf{Section 7:} Conclusion and Future Work.  
\end{itemize}

\subsection{Report Structure}
The remainder of this report is organized as follows. Section 2 provides background on the cryptographic principles underlying the project, including ECC, ECDSA, and hash functions. Section 3 details the high-level system architecture and design choices. Sections 4 and 5 present the specific implementation details of the \texttt{filecheck\_secure} tool and the secure logging suite, respectively. Section 6 offers a comprehensive developer's guide to the API and its usage. Section 7 provides a performance and security analysis of the implemented systems. Finally, Section 8 concludes the report with a summary of findings and suggestions for future work.

\newpage
\begin{figure}[h!]
	\centering
	\begin{adjustbox}{max width=\textwidth}
		\begin{forest}
			% Define styles for the tree
			for tree={
				draw,
				thick,
				rounded corners=3pt,
				font=\sffamily,
				edge={-Latex, thick},
				grow=east,         % Grow from left to right
				l sep=15mm,         % vertical separation between siblings
				s sep=5mm,          % horizontal separation between parent/child
				anchor=west,
				child anchor=west,
				align=center        % Center text in nodes
			}
			% Style for the root node
			[The Big Data Processing using ECC, fill=blue!20, font=\sffamily\bfseries
			% Level 1 Nodes (Chapters) - REORDERED TO RENDER CORRECTLY
			[Appendices \& References, fill=gray!20]
			[Chapter 8. Conclusion, fill=red!20
			[Summary of Results]
			[Limitations]
			[Future Work]
			]
			[Chapter 7. Analysis, fill=cyan!20
			[Performance Benchmarks]
			[Security Considerations]
			]
			[Chapter 6. Developer's Guide, fill=cyan!20
			[API Data Structures]
			[API Function Reference]
			[Sample Code]
			]
			[Chapter 5. Application 2: Secure Logging Suite, fill=purple!20
			[\texttt{logtool} Utility]
			[\texttt{logaggregator} Service]
			]
			[Chapter 4. Application 1: \texttt{filecheck\_secure}, fill=purple!20
			[Functional Modes]
			[Implementation Details]
			]
			[Chapter 3. Architecture, fill=orange!20
			[Core Components]
			[Design Goals]
			[Crypto Parameters]
			]
			[Chapter 2. Background, fill=orange!20
			[PKC \& ECC]
			[ECDSA \& Hashes]
			[\texttt{libecc} Overview]
			]
			[Chapter 1. Introduction, fill=green!20
			[Motivation \& Problem]
			[Scope \& Contributions]
			]
			]
		\end{forest}
	\end{adjustbox}
	\caption{A hierarchical overview of the structure of document.}
	\label{fig:report_structure_full}
\end{figure}