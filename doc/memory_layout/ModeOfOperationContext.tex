\begin{tikzpicture}[x=1cm, y=-0.07cm, font=\scriptsize, draw=black, >=Stealth, scale=.8]
	% BlockCipherContext outline
	\draw[thick] (0,0) rectangle ++(4,128);  % 128 bytes tall, 3 cm wide
	\draw[thick] (0,8) -- ++(4,0);          % line separating api vs internal_data at 8 bytes
	
	% BlockCipherContext labels and offsets
	\node[anchor=south, align=center, font=\small] at (2,0) {\textbf{ModeOfOperationContext}\\ \textbf{(800=8+792 bytes)}};
	\node[align=center] at (2,4) {\texttt{ModeOfOperationApi*}};
	\node[align=center] at (2,68) {\texttt{CipherInternal}\\ (union, 792 bytes)};
	
	\node[anchor=east] at (0,0) {{\tiny\ttfamily 0x000}};   % top address
	\node[anchor=east] at (0,8) {{\tiny\ttfamily 0x008}};
	\node[anchor=east] at (0,16) {{\tiny\ttfamily 0x010}};
	\node[anchor=east] at (0,24) {{\tiny\ttfamily 0x018}};
	\node[anchor=east] at (0,120) {{\tiny\ttfamily 0x318}};
	\node[anchor=east] at (0,128) {{\tiny\ttfamily 0x320}};
	
	% Expand union internal_data: AES, ARIA, LEA internal structures side-by-side
	% Base alignment line for union (offset 0x08 in context)
	\draw[densely dashed, color=gray!50] (4,8) -- ++(14,0) node[anchor=west]{\tiny\ttfamily 0x008};
	\draw[densely dashed, color=gray!50] (4,16) -- ++(14,0) node[anchor=west]{\tiny\ttfamily 0x010};
	\draw[densely dashed, color=gray!50] (4,24) -- ++(14,0) node[anchor=west]{\tiny\ttfamily 0x018};
	\draw[densely dashed, color=gray!50] (4,45) -- ++(14,0) node[anchor=west]{\tiny\ttfamily 0x108};
	\draw[densely dashed, color=gray!50] (4,53) -- ++(14,0) node[anchor=west]{\tiny\ttfamily 0x110};
	\draw[densely dashed, color=gray!50] (4,75) -- ++(14,0) node[anchor=west]{\tiny\ttfamily 0x128};
	\draw[densely dashed, color=gray!50] (4,83) -- ++(14,0) node[anchor=west]{\tiny\ttfamily 0x130};
	\draw[densely dashed, color=gray!50] (4,120) -- ++(14,0) node[anchor=west]{\tiny\ttfamily 0x318};
	\draw[densely dashed, color=gray!50] (4,128) -- ++(14,0) node[anchor=west]{\tiny\ttfamily 0x320};
	
	% AES internal struct (80 bytes used out of 120)
	\draw (5,8) rectangle ++(3,120);       % union size tall (for visual alignment)
	\draw (5,16) -- ++(3,0);
	\draw (5,24) -- ++(3,0);
	\draw (5,45) -- ++(3,0);
	\draw (5,53) -- ++(3,0);
	\node[anchor=south, align=center] at (6.5,8) {\textbf{aes\_internal}\\ \textbf{(8+8+240+8=264 bytes)}};
	\node at (6.5,12) {\texttt{size\_t block\_size}};
	\node at (6.5,20) {\texttt{size\_t key\_len}};
	\node[align=center] at (6.5,36) {\texttt{u32 round\_keys[60]} \\ (240 bytes)};
	\node at (6.5,49) {\texttt{int nr}\; (4 bytes)};
	\node[color=gray, align=center] at (6.5,90) {\footnotesize (unused\\ \footnotesize 528 bytes)};
	\filldraw[gray!70, opacity=.2] (5, 53) rectangle (8, 128);
	
	\filldraw[teal!70, opacity=.2] (5, 8) rectangle (8, 16);
	\filldraw[orange!70, opacity=.2] (5, 16) rectangle (8, 24);
	\filldraw[red!70, opacity=.2] (5, 24) rectangle (8, 45);
	\filldraw[blue!70, opacity=.2] (5, 45) rectangle (8, 53);
	
	% ARIA internal struct (same as AES, 80 bytes)
	\draw (9,8) rectangle ++(3,120);
	\draw (9,16) -- ++(3,0);
	\draw (9,24) -- ++(3,0);
	\draw (9,75) -- ++(3,0);
	\draw (9,83) -- ++(3,0);
	\node[anchor=south, align=center] at (10.5,8) {\textbf{aria\_internal}\\ \textbf{(8+8+272+8=296 bytes)}};
	\node at (10.5,12) {\texttt{size\_t block\_size}};
	\node at (10.5,20) {\texttt{size\_t key\_len}};
	\node[align=center] at (10.5,50) {\texttt{u32 round\_keys[68]} \\ (272 bytes)};
	\node at (10.5,79) {\texttt{int nr}\; (4 bytes)};
	\node[color=gray, align=center] at (10.5,104) {\footnotesize (unused\\ \footnotesize 496 bytes)};
	\filldraw[gray!70, opacity=.2] (9, 83) rectangle (12, 128);
	
	\filldraw[teal!70, opacity=.2] (9, 8) rectangle (12, 16);
	\filldraw[orange!70, opacity=.2] (9, 16) rectangle (12, 24);
	\filldraw[red!70, opacity=.2] (9, 24) rectangle (12, 75);
	\filldraw[blue!70, opacity=.2] (9, 75) rectangle (12, 83);
	
	% LEA internal struct (116 bytes + 4 bytes padding)
	\draw (13,8) rectangle ++(3,120);
	\draw (13,16) -- ++(3,0);
	\draw (13,24) -- ++(3,0);
	\draw (13,120) -- ++(3,0);
	\node[anchor=south, align=center] at (14.5,8) {\textbf{lea\_internal}\\ \textbf{(8+8+768+8=792 bytes)}};
	\node at (14.5,12) {\texttt{size\_t block\_size}};
	\node at (14.5,20) {\texttt{size\_t key\_len}};
	\node[align=center] at (14.5,64) {\texttt{u32 round\_keys[192]} \\ (768 bytes)};
	\node at (14.5,124) {\texttt{int nr}\; (4 bytes)};
	
	\filldraw[teal!70, opacity=.2] (13, 8) rectangle (16, 16);
	\filldraw[orange!70, opacity=.2] (13, 16) rectangle (16, 24);
	\filldraw[red!70, opacity=.2] (13, 24) rectangle (16, 120);
	\filldraw[blue!70, opacity=.2] (13, 120) rectangle (16, 128);
	
	% Brace to indicate union grouping
	\draw[decorate,decoration={brace,amplitude=4pt}] (16,130) -- node[below=4pt]{Union members (share same memory)} (5,130);
	
	% BlockCipherApi vtable structure (40 bytes)
	% Place vtable below (starting at y=140 for offset 0)
	\filldraw[blue!50, opacity=.8] (0, 160) rectangle (3, 168);
	\draw[thick] (0,160) rectangle ++(3,32);
	\draw[thick] (0,168) -- ++(3,0);
	\draw[thick] (0,176) -- ++(3,0);
	\draw[thick] (0,184) -- ++(3,0);
	%	\draw[thick] (0,192) -- ++(3,0);
	\node[anchor=south,align=center] at (1.5,160) {\textbf{BlockCipherApi (vtable)}\\ \textbf{8$\times$4=32 bytes}};
	\node at (1.5,164) {\texttt{const char* name}};
	\node at (1.5,172) {\texttt{void*}};
	\node at (1.5,180) {\texttt{void*}};
	%	\node at (1.5,188) {\texttt{void*}};
	\node at (1.5,188) {\texttt{void*}};
	
	% Arrows: BlockCipherContext.api -> vtable, and vtable function ptrs -> functions
	% API pointer arrow (from BCC.api field to vtable)
	\draw[->, thick] (0,4) -| (-1,4) -- (-1,164) -- ++(1,0);
	
	% Function implementation nodes for AES
	\node[draw, anchor=west, fill=magenta!10] (fn_init)        at (5,172) {\texttt{aes\_init(BlockCipherContext* ctx, /* ... */ );}};
	\node[draw, anchor=west, fill=green!10] (fn_encrypt)     at (5,180) {\texttt{aes\_process\_block(BlockCipherContext* ctx, /* ... */ );}};
	%	\node[draw, anchor=west, fill=lime!10] (fn_decrypt)     at (3.5,188) {\texttt{decrypt(BlockCipherContext* ctx, const u8* ct, u8* pt);}};
	\node[draw, anchor=west, fill=cyan!10] (fn_dispose)     at (5,188) {\texttt{dispose(BlockCipherContext* ctx);}};
	
	\draw[->] (3,172) -- (fn_init.west);
	\draw[->] (3,180) -- (fn_encrypt.west);
	%	\draw[->] (3,188) -- (fn_decrypt.west);
	\draw[->] (3,188) -- (fn_dispose.west);
\end{tikzpicture}